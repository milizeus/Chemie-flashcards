%%%%%%%%%%%%%%%%%%%%%%%%%%%%%%%%%%%%%%%%%%%%%%%%%%%%%%%%%%%%%%%%%%%%%%%%%%
%%%%%%%%%%%%%%%%%%%%%%%%%%%%%%%%%%%%%%%%%%%%%%%%%%%%%%%%%%%%%%%%%%%%%%%%%%
\section*{ALLGEMEINE FRAGEN}
%%%%%%%%%%%%%%%%%%%%%%%%%%%%%%%%%%%%%%%%%%%%%%%%%%%%%%%%%%%%%%%%%%%%%%%%%%
%%%%%%%%%%%%%%%%%%%%%%%%%%%%%%%%%%%%%%%%%%%%%%%%%%%%%%%%%%%%%%%%%%%%%%%%%%

%%%%%%%%%%%%%%%%%%%%%%%%%%%%%%%%%%%%%%
%\begin{karte}{
%	%Frage
%	Wie ist das Periodensystem aufgebaut und welche Informationen kann man daraus ablesen.
%
%	}
%	%Antwort
%	Antwort
%
%\end{karte}
%%%%%%%%%%%%%%%%%%%%%%%%%%%%%%%%%%%%%%

%%%%%%%%%%%%%%%%%%%%%%%%%%%%%%%%%%%%%
\begin{karte}{
	
	Welche chemischen Bindungen kennen Sie? Geben Sie zu jeder ein Beispiel und erkl�ren Sie.
	
	}
	
	Antwort
	
\end{karte}
%%%%%%%%%%%%%%%%%%%%%%%%%%%%%%%%%%%%%

%%%%%%%%%%%%%%%%%%%%%%%%%%%%%%%%%%%%%
\begin{karte}{
	
	Wie h�ngen die 1. Ionisierungsenergie und Elektronenaffinit�t mit der Lage im Periodensystem zusammen?
	
	}
	
	Antwort
	
\end{karte}
%%%%%%%%%%%%%%%%%%%%%%%%%%%%%%%%%%%%%

%%%%%%%%%%%%%%%%%%%%%%%%%%%%%%%%%%%%%
\begin{karte}{
	
	Welche Arten der chemischen Formelschreibweise kennen Sie? Geben sie jeweils ein Beispiel an.
	
	}
	
	Antwort
	
\end{karte}
%%%%%%%%%%%%%%%%%%%%%%%%%%%%%%%%%%%%%

%%%%%%%%%%%%%%%%%%%%%%%%%%%%%%%%%%%%%
\begin{karte}{
	
	Unter welchen Bedingungen kommt es zu einer sigma- und unter welchen Bedingungen zu einer pi \textendash{} Bindung? Erkl�ren Sie anhand eines Beispiels.
	
	}
	
	Antwort
	
\end{karte}
%%%%%%%%%%%%%%%%%%%%%%%%%%%%%%%%%%%%%

%%%%%%%%%%%%%%%%%%%%%%%%%%%%%%%%%%%%%
\begin{karte}{
	
	Welche Gasgesetze kennen Sie und welche Gr��en bringen sie in Zusammenhang?
	
	}
	
	Antwort
	
\end{karte}
%%%%%%%%%%%%%%%%%%%%%%%%%%%%%%%%%%%%%

%%%%%%%%%%%%%%%%%%%%%%%%%%%%%%%%%%%%%
\begin{karte}{
	
	Nennen Sie die wichtigsten Kennzeichen von Polymeren. Nennen Sie drei und geben ihre Verwendung an.
	
	}
	
	Antwort
	
\end{karte}
%%%%%%%%%%%%%%%%%%%%%%%%%%%%%%%%%%%%%

%%%%%%%%%%%%%%%%%%%%%%%%%%%%%%%%%%%%%
\begin{karte}{
	
	Was bedeutet L�slichkeit und wovon h�ngt die L�slichkeit eines Stoffes ab?
	
	}
	
	Antwort
	
\end{karte}
%%%%%%%%%%%%%%%%%%%%%%%%%%%%%%%%%%%%%}

%%%%%%%%%%%%%%%%%%%%%%%%%%%%%%%%%%%%%
\begin{karte}{
	
	Welche Konzentrationsangaben kennen Sie und wie sind sie jeweils definiert?
	
	}
	
	Antwort
	
\end{karte}
%%%%%%%%%%%%%%%%%%%%%%%%%%%%%%%%%%%%%

%%%%%%%%%%%%%%%%%%%%%%%%%%%%%%%%%%%%%
\begin{karte}{
	
	Welche kolligativen Eigenschaften kennen Sie und was wissen Sie dar�ber?
	
	}
	
	Antwort
	
\end{karte}
%%%%%%%%%%%%%%%%%%%%%%%%%%%%%%%%%%%%%

%%%%%%%%%%%%%%%%%%%%%%%%%%%%%%%%%%%%%
\begin{karte}{
	
	Wovon ist die Reaktionsgeschwindigkeit abh�ngig und was bedeutet Katalyse?
	
	}
	
	Antwort
	
\end{karte}
%%%%%%%%%%%%%%%%%%%%%%%%%%%%%%%%%%%%%

%%%%%%%%%%%%%%%%%%%%%%%%%%%%%%%%%%%%%
\begin{karte}{
	
	Erkl�ren Sie den Inhalt der 3 Haupts�tze der Thermodynamik.
	
	}
	
	Antwort
	
\end{karte}
%%%%%%%%%%%%%%%%%%%%%%%%%%%%%%%%%%%%%

%%%%%%%%%%%%%%%%%%%%%%%%%%%%%%%%%%%%%
\begin{karte}{
	
	Erkl�ren Sie die Begriffe Aminos�uren, Peptid und Protein.
	
	}
	
	Antwort
	
\end{karte}
%%%%%%%%%%%%%%%%%%%%%%%%%%%%%%%%%%%%%

%%%%%%%%%%%%%%%%%%%%%%%%%%%%%%%%%%%%%
\begin{karte}{
	
	Erkl�ren Sie die Begriffe Kohlenhydrat, Monosaccharid, Disaccharid und Polysaccharid und geben sie jeweils ein Beipiel an.
	
	}
	
	Antwort
	
\end{karte}
%%%%%%%%%%%%%%%%%%%%%%%%%%%%%%%%%%%%%

%%%%%%%%%%%%%%%%%%%%%%%%%%%%%%%%%%%%%
\begin{karte}{
	
	Erkl�ren Sie die Begriffe DNA, RNA, Nukleins�ure, Nukleotid.
	
	}
	
	Antwort
	
\end{karte}
%%%%%%%%%%%%%%%%%%%%%%%%%%%%%%%%%%%%%





