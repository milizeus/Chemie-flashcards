%!TEX root = ../chemie.tex




\newpage

\chapter{ALLGEMEINE FRAGEN}

\begin{enumerate}

	%%%%%%%%%%%%%%%%%%%%%%%%%%%%%%%%%%%%%%%%%%%%%%%%%%%%%%%%

	\item Wie ist das Periodensystem aufgebaut und welche Informationen kann
	man daraus ablesen.
	
	\begin{enumerate}
		\item a
	\end{enumerate}

	%%%%%%%%%%%%%%%%%%%%%%%%%%%%%%%%%%%%%%%%%%%%%%%%%%%%%%%%
	\item Welche chemischen Bindungen kennen Sie? Geben Sie zu jeder ein Beispiel
	und erkl�ren Sie.
	
	\begin{enumerate}
		\item a
	\end{enumerate}

	%%%%%%%%%%%%%%%%%%%%%%%%%%%%%%%%%%%%%%%%%%%%%%%%%%%%%%%%

	\item Wie h�ngen die 1. Ionisierungsenergie und Elektronenaffinit�t mit
	der Lage im Periodensystem zusammen?
	
	\begin{enumerate}
		\item a
	\end{enumerate}

	%%%%%%%%%%%%%%%%%%%%%%%%%%%%%%%%%%%%%%%%%%%%%%%%%%%%%%%%

	\item Welche Arten der chemischen Formelschreibweise kennen Sie? Geben sie
	jeweils ein Beispiel an.
	
	\begin{enumerate}
		\item a
	\end{enumerate}

	%%%%%%%%%%%%%%%%%%%%%%%%%%%%%%%%%%%%%%%%%%%%%%%%%%%%%%%%

	\item Unter welchen Bedingungen kommt es zu einer sigma- und unter welchen
	Bedingungen zu einer pi \textendash{} Bindung? Erkl�ren Sie anhand
	eines Beispiels.
	
	\begin{enumerate}
		\item a
	\end{enumerate}

	%%%%%%%%%%%%%%%%%%%%%%%%%%%%%%%%%%%%%%%%%%%%%%%%%%%%%%%%

	\item Welche Gasgesetze kennen Sie und welche Gr��en bringen sie in Zusammenhang?
	
	\begin{enumerate}
		\item a
	\end{enumerate}

	%%%%%%%%%%%%%%%%%%%%%%%%%%%%%%%%%%%%%%%%%%%%%%%%%%%%%%%%

	\item Nennen Sie die wichtigsten Kennzeichen von Polymeren. Nennen Sie drei
	und geben ihre Verwendung an.
	
	\begin{enumerate}
		\item a
	\end{enumerate}

	%%%%%%%%%%%%%%%%%%%%%%%%%%%%%%%%%%%%%%%%%%%%%%%%%%%%%%%%

	\item Was bedeutet L�slichkeit und wovon h�ngt die L�slichkeit eines Stoffes
	ab?
	
	\begin{enumerate}
		\item a
	\end{enumerate}

	%%%%%%%%%%%%%%%%%%%%%%%%%%%%%%%%%%%%%%%%%%%%%%%%%%%%%%%%

	\item Welche Konzentrationsangaben kennen Sie und wie sind sie jeweils definiert?
	
	\begin{enumerate}
		\item a
	\end{enumerate}

	%%%%%%%%%%%%%%%%%%%%%%%%%%%%%%%%%%%%%%%%%%%%%%%%%%%%%%%%

	\item Welche kolligativen Eigenschaften kennen Sie und was wissen Sie dar�ber?
	
	\begin{enumerate}
		\item a
	\end{enumerate}

	%%%%%%%%%%%%%%%%%%%%%%%%%%%%%%%%%%%%%%%%%%%%%%%%%%%%%%%%

	\item Wovon ist die Reaktionsgeschwindigkeit abh�ngig und was bedeutet Katalyse?
	
	\begin{enumerate}
		\item a
	\end{enumerate}

	%%%%%%%%%%%%%%%%%%%%%%%%%%%%%%%%%%%%%%%%%%%%%%%%%%%%%%%%

	\item Erkl�ren Sie den Verlauf einer S�ure-Base Titration.
	
	\begin{enumerate}
		\item a
	\end{enumerate}

	%%%%%%%%%%%%%%%%%%%%%%%%%%%%%%%%%%%%%%%%%%%%%%%%%%%%%%%%

	\item Erkl�ren Sie den Inhalt der 3 Haupts�tze der Thermodynamik.
	
	\begin{enumerate}
		\item a
	\end{enumerate}

	%%%%%%%%%%%%%%%%%%%%%%%%%%%%%%%%%%%%%%%%%%%%%%%%%%%%%%%%

	\item Erkl�ren Sie die Begriffe Aminos�uren, Peptid und Protein.
	
	\begin{enumerate}
		\item a
	\end{enumerate}

	%%%%%%%%%%%%%%%%%%%%%%%%%%%%%%%%%%%%%%%%%%%%%%%%%%%%%%%%

	\item Erkl�ren Sie die Begriffe Kohlenhydrat, Monosaccharid, Disaccharid
	und Polysaccharid und geben sie jeweils ein Beipiel an.
	
	\begin{enumerate}
		\item a
	\end{enumerate}

	%%%%%%%%%%%%%%%%%%%%%%%%%%%%%%%%%%%%%%%%%%%%%%%%%%%%%%%%

	\item Erkl�ren Sie die Begriffe DNA, RNA, Nukleins�ure, Nukleotid.
	
	\begin{enumerate}
		\item a
	\end{enumerate}

	%%%%%%%%%%%%%%%%%%%%%%%%%%%%%%%%%%%%%%%%%%%%%%%%%%%%%%%%

\end{enumerate}



