%%%%%%%%%%%%%%%%%%%%%%%%%%%%%%%%%%%%%%%%%%%%%%%%%%%%%%%%%%%%%%%%%%%%%%%%%%
%%%%%%%%%%%%%%%%%%%%%%%%%%%%%%%%%%%%%%%%%%%%%%%%%%%%%%%%%%%%%%%%%%%%%%%%%%
\section*{GASE \& FL�SSIGKEITEN \& L�SUNGEN}
%%%%%%%%%%%%%%%%%%%%%%%%%%%%%%%%%%%%%%%%%%%%%%%%%%%%%%%%%%%%%%%%%%%%%%%%%%
%%%%%%%%%%%%%%%%%%%%%%%%%%%%%%%%%%%%%%%%%%%%%%%%%%%%%%%%%%%%%%%%%%%%%%%%%%

%%%%%%%%%%%%%%%%%%%%%%%%%%%%%%%%%%%%%
	\begin{karte}{
		Eine Gasmischung aus 6.00 g \ce{O2} und 9.00 g \ce{CH4} wird bei 0�C in einen Beh�lter (V = 100 mL) gegeben. \\
		Wie sind die Partialdr�cke f�r jedes Gas und wie ist der Gesamtdruck im Beh�lter in atm? \\
		$R = 0.0821  \frac{L\,atm}{Mol \,K}$
		}
		
		\begin{enumerate}
			
				\item 
				$M_{\ce{O2}}		=2\cdot16								=32\frac{g}{Mol}$\\
				$M_{\ce{CH4}}	=12+4\cdot1							=16\frac{g}{Mol}$\\
				$n_{\ce{O2}}		=\frac{6g}{32\frac{g}{Mol}}		=0,1875 Mol$\\
				$n_{\ce{CH4}}	=\frac{9g}{16\frac{g}{Mol}}		=0,5625 Mol$
				
				$\displaystyle pV=nRT \Rightarrow p=\frac{nRT}{V}$\\
						
				$\displaystyle P_{\ce{O2}}		=\frac{0,1875Mol\cdot0,0821\frac{L\,atm}{Mol \,K}\cdot273,15K}{0,1L}	=42atm$\\
				$\displaystyle P_{\ce{CH4}}	=\frac{0,5625Mol\cdot0,0821\frac{L\,atm}{Mol \,K}\cdot273,15K}{0,1L}	=126atm$
		
				$\displaystyle p_{ges} = \sum_{i=1}^k p_i$
				
				$P_{ges}					=42atm + 126atm					=168atm$
				
			\end{enumerate}
		
	\end{karte}
%%%%%%%%%%%%%%%%%%%%%%%%%%%%%%%%%%%%%

%%%%%%%%%%%%%%%%%%%%%%%%%%%%%%%%%%%%%
	\begin{karte}{
		Ammoniumnitrit zersetzt sich beim Erhitzen zu \ce{N2} Gas: \ce{NH4NO2 -> N_{2(g)} + 2H2O_{(l)}} 
		Wenn eine Probe in einem Reagenzglas zersetzt wird, werden 511 mL \ce{N2}-Gas �ber Wasser bei 26�C und 745 Torr Gesamtdruck aufgefangen. \\
		
		Wie viel Gramm Ammoniumnitrit wurden zersetzt? \\
		$R = 62,36 \frac{L\,torr}{Mol \,K}$
		}
		
		\begin{enumerate}
		
				\item 
				$M_{\ce{NH4NO2}}		=14+4\cdot1+14+2\cdot16	=64\frac{g}{Mol}$ 
				
				$\displaystyle n			=\frac{P\cdot V}{R\cdot T}		
									=\frac{745Torr \cdot 0,511 L}{62,36  \frac{L\,torr}{Mol \,K} \cdot (273,15+26)K}
									=0,02Mol$
				
				$m_{\ce{NH4NO2}}		=64\frac{g}{Mol} \cdot 0,02Mol		=1,28g$
				
			\end{enumerate}
		
	\end{karte}
%%%%%%%%%%%%%%%%%%%%%%%%%%%%%%%%%%%%%

%%%%%%%%%%%%%%%%%%%%%%%%%%%%%%%%%%%%%
	\begin{karte}{
		Cyclopropan, besteht aus 85.7 Massen\% \ce{C} und 14.3 Massen\% \ce{H}. 
			Wenn 1.56 g Cyclopropan ein Volumen von 1 L bei 0.984 atm und 50�C hat, 
			wie ist dann die Molek�lformel von Cyclopropan? \\
			W�rden Sie erwarten, dass Cyclopropan mehr oder weniger als Argon vom idealen Gasverhalten bei moderaten Dr�cken und Zimmertemperatur abweicht? \\
			Erkl�ren Sie! \\
			$R = 0.0821  \frac{L\,atm}{Mol \,K}$
		}
		
		\begin{enumerate}
			
				\item
				$M_{\ce{C}}	=12 \frac{g}{Mol}$ \\
				$M_{\ce{H}}	=1 \frac{g}{Mol}$ \\
			
			\end{enumerate}
		
	\end{karte}
%%%%%%%%%%%%%%%%%%%%%%%%%%%%%%%%%%%%%

%%%%%%%%%%%%%%%%%%%%%%%%%%%%%%%%%%%%%
	\begin{karte}{
		9.23 g einer Mischung von Magnesiumcarbonat und Calciumoxid wird mit einem �berschuss von Salzs�ure behandelt. Die resultierende Reaktion erzeugt 1.72 L Kohlendioxid bei 28�C und 743 Torr. \\
		Schreiben Sie ausgeglichene chemische Gleichungen f�r die Reaktionen, die zwischen der Salzs�ure und jedem Bestandteil der Mischung auftreten. \\
		Berechnen sie die Gesamtmolzahl von Kohlendioxid, die durch diese Reaktion gebildet wird. \\
		Unter der Annahme, dass die Reaktionen vollst�ndig ablaufen, berechnen sie die Massenprozent von Magnesiumcarbonat in der Mischung.\\
		(R = 62.36 L torr/Mol K)
		}
		
	\end{karte}
%%%%%%%%%%%%%%%%%%%%%%%%%%%%%%%%%%%%%

%%%%%%%%%%%%%%%%%%%%%%%%%%%%%%%%%%%%%
	\begin{karte}{
		Zeichnen und beschreiben Sie das Phasendiagramm von Wasser. \\
		Definieren Sie die beiden besonderen Punkte.
		}
		
	\end{karte}
%%%%%%%%%%%%%%%%%%%%%%%%%%%%%%%%%%%%%

%%%%%%%%%%%%%%%%%%%%%%%%%%%%%%%%%%%%%
	\begin{karte}{
		Welche Art von Anziehungskr�ften liegt zwischen Teilchen in \\
			a) molekularen Kristallen, \\
			b) kovalenten Kristallen, \\
			c) ionischen Kristallen und\\
			d) metallischen Kristallen vor?
		}
		
	\end{karte}
%%%%%%%%%%%%%%%%%%%%%%%%%%%%%%%%%%%%%

%%%%%%%%%%%%%%%%%%%%%%%%%%%%%%%%%%%%%
	\begin{karte}{
		Wie unterscheidet ein amorpher Festk�rper sich von einem kristallinen?\\
		Geben Sie je ein Beispiel f�r einen amorphen und einen kristallinen Festk�rper.
		}
		
	\end{karte}
%%%%%%%%%%%%%%%%%%%%%%%%%%%%%%%%%%%%%

%%%%%%%%%%%%%%%%%%%%%%%%%%%%%%%%%%%%%
	\begin{karte}{
		Glycerin ist ein wasserl�slicher Nichtelektrolyt mit einer Dichte
			von 1.26 g/mL bei 25�C. Berechnen sie den Dampfdruck einer
			L�sung, die durch Zugabe von 50 mL Glycerin zu 500 mL Wasser hergestellt
			wird. Der Dampfdruck von reinem Wasser bei 25�C betr�gt
			23.8 Torr.
		}
		
	\end{karte}
%%%%%%%%%%%%%%%%%%%%%%%%%%%%%%%%%%%%%

%%%%%%%%%%%%%%%%%%%%%%%%%%%%%%%%%%%%%
	\begin{karte}{
		Der Dampfdruck von reinem Wasser bei 110�C ist 1070 Torr.
			Eine L�sung aus Ethylenglykol und Wasser hat einen Dampfdruck von
			1.00 atm bei 110�C. Berechnen Sie den Molenbruch von Ethylenglykol.
		}
		
	\end{karte}
%%%%%%%%%%%%%%%%%%%%%%%%%%%%%%%%%%%%%

%%%%%%%%%%%%%%%%%%%%%%%%%%%%%%%%%%%%%
	\begin{karte}{
		Wenn 10.0 g \ce{Hg(NO3)2} in 1 kg Wasser aufgel�st wird, ist der
			Gefrierpunkt der L�sung -0.162 �C. Wenn 10.0 g \ce{HgCl2}
			in 1 kg Wasser gel�st werden, gefriert die L�sung bei -0.0685�C.
			Bestimmen sie anhand der dieser Daten, welches der st�rkere Elektrolyt
			ist und berechnen sie die Siedepunktserh�hung in beiden F�llen. ($K_{b\,\ce{H2O}}$
			= 0.51 �C/m)
		}
		
	\end{karte}
%%%%%%%%%%%%%%%%%%%%%%%%%%%%%%%%%%%%%


