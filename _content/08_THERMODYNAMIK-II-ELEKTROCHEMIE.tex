%!TEX root = ../chemie.tex



\newpage

\chapter{THERMODYNAMIK II \& ELEKTROCHEMIE}

	%%%%%%%%%%%%%%%%%%%%%%%%%%%%%%%%%%%%%%%%%%%%%%%%%%%%%%%%
	
	\begin{enumerate}
	\item Sagen sie voraus, ob $\Delta$S f�r folgende Prozesse positiv
	oder negativ ist, wobei wir davon ausgehen, dass alle bei konstanter
	Temperatur ablaufen:\\
	(a) \ce{H2O_{(g)} -> H2O_{(l)}}\\
	(b) \ce{Ag_{(aq)}^{+} + Cl_{(aq)}^{-} -> AgCl_{(s)}}\\
	(c) \ce{4Fe_{(s)} + 3O2_{(g)} -> 2Fe2O3_{(s)}}\\
	(d) \ce{N2_{(g)} + O2_{(g)} -> 2NO_{(g)}}
	
	\begin{enumerate}
		\item a
	\end{enumerate}

	%%%%%%%%%%%%%%%%%%%%%%%%%%%%%%%%%%%%%%%%%%%%%%%%%%%%%%%%

	
	\item (a) Was ist das Besondere an einem reversiblen Prozess? \\
	(b) Gehen Sie davon aus, dass ein reversibler Prozess umgekehrt wird, und das
	System in seinen Ausgangszustand zur�ckversetzt wird. Was l�sst sich
	�ber die Umgebung nach der Umkehrung des Prozesses aussagen? \\
	(c) Unter welchen Umst�nden handelt es sich beim Verdampfen von Wasser zu Dampf
	um einen reversiblen Prozess?
	
	\begin{enumerate}
		\item a
	\end{enumerate}

	%%%%%%%%%%%%%%%%%%%%%%%%%%%%%%%%%%%%%%%%%%%%%%%%%%%%%%%%

	\item Vervollst�ndigen Sie folgende Redoxgleichung: \\
	\ce{MnO- + Fe2+ + H+ -> MnO2 + Fe^{3+} + H2O}
	
	\begin{enumerate}
		\item a
	\end{enumerate}

	%%%%%%%%%%%%%%%%%%%%%%%%%%%%%%%%%%%%%%%%%%%%%%%%%%%%%%%%

	\item Vervollst�ndigen Sie folgende Redoxgleichung:\\
	\ce{MnO4- + Mn^{2+} + H+ -> MnO2 + H2O}
	
	\begin{enumerate}
		\item a
	\end{enumerate}

	%%%%%%%%%%%%%%%%%%%%%%%%%%%%%%%%%%%%%%%%%%%%%%%%%%%%%%%%

	\item Vervollst�ndigen Sie folgende Redoxgleichung: \\
	\ce{Cr2O7^{2-} + CH3OH + H+ -> Cr^{3+} + CO2 + H2O}
	
	\begin{enumerate}
		\item a
	\end{enumerate}

	%%%%%%%%%%%%%%%%%%%%%%%%%%%%%%%%%%%%%%%%%%%%%%%%%%%%%%%%

	\item Beschreiben sie mit Hilfe einer Skizze den Aufbau einer Alkalibatterie
	und Formulieren Sie die Anoden und Kathodenreaktion.
	
	\begin{enumerate}
		\item a
	\end{enumerate}

	%%%%%%%%%%%%%%%%%%%%%%%%%%%%%%%%%%%%%%%%%%%%%%%%%%%%%%%%

	\item Beschreiben sie mit Hilfe einer Skizze die Korrosion von Eisen und
	Formulieren Sie die Anoden und Kathodenreaktion, sowie die Gesamtreaktion.
	
	\begin{enumerate}
		\item a
	\end{enumerate}
	
	%%%%%%%%%%%%%%%%%%%%%%%%%%%%%%%%%%%%%%%%%%%%%%%%%%%%%%%%
	
\end{enumerate}



