%%%%%%%%%%%%%%%%%%%%%%%%%%%%%%%%%%%%%%%%%%%%%%%%%%%%%%%%%%%%%%%%%%%%%%%%%%
%%%%%%%%%%%%%%%%%%%%%%%%%%%%%%%%%%%%%%%%%%%%%%%%%%%%%%%%%%%%%%%%%%%%%%%%%%
\section*{THERMODYNAMIK II \& ELEKTROCHEMIE}
%%%%%%%%%%%%%%%%%%%%%%%%%%%%%%%%%%%%%%%%%%%%%%%%%%%%%%%%%%%%%%%%%%%%%%%%%%
%%%%%%%%%%%%%%%%%%%%%%%%%%%%%%%%%%%%%%%%%%%%%%%%%%%%%%%%%%%%%%%%%%%%%%%%%%

%%%%%%%%%%%%%%%%%%%%%%%%%%%%%%%%%%%%%
	\begin{karte}{
		
		Sagen sie voraus, ob $\Delta$S f�r folgende Prozesse positiv oder negativ ist, wobei wir davon ausgehen, dass alle bei konstanter Temperatur ablaufen:\\
		(a) \ce{H2O_{(g)} -> H2O_{(l)}}\\
		(b) \ce{Ag_{(aq)}^{+} + Cl_{(aq)}^{-} -> AgCl_{(s)}}\\
		(c) \ce{4Fe_{(s)} + 3O2_{(g)} -> 2Fe2O3_{(s)}}\\
		(d) \ce{N2_{(g)} + O2_{(g)} -> 2NO_{(g)}}
		
		}
		
	\end{karte}
%%%%%%%%%%%%%%%%%%%%%%%%%%%%%%%%%%%%%

%%%%%%%%%%%%%%%%%%%%%%%%%%%%%%%%%%%%%
	\begin{karte}{
		
		(a) Was ist das Besondere an einem reversiblen Prozess? \\
		(b) Gehen Sie davon aus, dass ein reversibler Prozess umgekehrt wird, und das System in seinen Ausgangszustand zur�ckversetzt wird. Was l�sst sich �ber die Umgebung nach der Umkehrung des Prozesses aussagen? \\
		(c) Unter welchen Umst�nden handelt es sich beim Verdampfen von Wasser zu Dampf um einen reversiblen Prozess?
		
		}
		
	\end{karte}
%%%%%%%%%%%%%%%%%%%%%%%%%%%%%%%%%%%%%

%%%%%%%%%%%%%%%%%%%%%%%%%%%%%%%%%%%%%
	\begin{karte}{
		
		Vervollst�ndigen Sie folgende Redoxgleichung: \\
		\ce{MnO- + Fe2+ + H+ -> MnO2 + Fe^{3+} + H2O}
		
		}
		
	\end{karte}
%%%%%%%%%%%%%%%%%%%%%%%%%%%%%%%%%%%%%

%%%%%%%%%%%%%%%%%%%%%%%%%%%%%%%%%%%%%
	\begin{karte}{
	
		Vervollst�ndigen Sie folgende Redoxgleichung:\\
		\ce{MnO4- + Mn^{2+} + H+ -> MnO2 + H2O}
		
		}
		
	\end{karte}
%%%%%%%%%%%%%%%%%%%%%%%%%%%%%%%%%%%%%

%%%%%%%%%%%%%%%%%%%%%%%%%%%%%%%%%%%%%
	\begin{karte}{
		
		Vervollst�ndigen Sie folgende Redoxgleichung: \\
		\ce{Cr2O7^{2-} + CH3OH + H+ -> Cr^{3+} + CO2 + H2O}
		
		}
		
	\end{karte}
%%%%%%%%%%%%%%%%%%%%%%%%%%%%%%%%%%%%%

%%%%%%%%%%%%%%%%%%%%%%%%%%%%%%%%%%%%%
	\begin{karte}{
		
		Beschreiben sie mit Hilfe einer Skizze den Aufbau einer Alkalibatterie und Formulieren Sie die Anoden und Kathodenreaktion.
		
		}
		
	\end{karte}
%%%%%%%%%%%%%%%%%%%%%%%%%%%%%%%%%%%%%

%%%%%%%%%%%%%%%%%%%%%%%%%%%%%%%%%%%%%
	\begin{karte}{
		
		Beschreiben sie mit Hilfe einer Skizze die Korrosion von Eisen und formulieren Sie die Anoden und Kathodenreaktion, sowie die Gesamtreaktion.
		
		}
		
	\end{karte}
%%%%%%%%%%%%%%%%%%%%%%%%%%%%%%%%%%%%%


