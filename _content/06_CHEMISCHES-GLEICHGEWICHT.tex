%%%%%%%%%%%%%%%%%%%%%%%%%%%%%%%%%%%%%%%%%%%%%%%%%%%%%%%%%%%%%%%%%%%%%%%%%%
%%%%%%%%%%%%%%%%%%%%%%%%%%%%%%%%%%%%%%%%%%%%%%%%%%%%%%%%%%%%%%%%%%%%%%%%%%
\section*{CHEMISCHES GLEICHGEWICHT}
%%%%%%%%%%%%%%%%%%%%%%%%%%%%%%%%%%%%%%%%%%%%%%%%%%%%%%%%%%%%%%%%%%%%%%%%%%
%%%%%%%%%%%%%%%%%%%%%%%%%%%%%%%%%%%%%%%%%%%%%%%%%%%%%%%%%%%%%%%%%%%%%%%%%%

%%%%%%%%%%%%%%%%%%%%%%%%%%%%%%%%%%%%%
	\begin{karte}{
		
		Bestimmen Sie mit folgenden Informationen:\\
		\ce{HF <-> H+ + F-} $K_{c}=6.8 \cdot 10^{-4}$ und\\
		\ce{H2C2O4 <-> 2H+ + C2O4^{2-}} $K_{c}=3,8 \cdot 10^{-6}$\\
		die Gleichgewichtskonstante $K_{c}$ der Reaktion  \ce{2HF + C2O4^{2-} <-> 2F- + H2C2O4} (alle Reaktionspartner sind aquatisiert). Zeichnen sie eine plausible Lewis-Strukturformel von \ce{H2C2O4}.
		
		}
		
	\end{karte}
%%%%%%%%%%%%%%%%%%%%%%%%%%%%%%%%%%%%%

%%%%%%%%%%%%%%%%%%%%%%%%%%%%%%%%%%%%%
	\begin{karte}{
		
		Die Gleichgewichtskonstanten $K_{p}$ (bei 700�C) f�r die Reaktionen:\\
		\ce{H2 +I2 <-> 2HI} $K_{p}=54.0$,\\
		\ce{N2 + 3H2 <-> 2NH3} $K_{p}=1,04\cdot10^{-4}$ \\
		sind gegeben. Bestimmen Sie den Wert f�r $K_{p}$ f�r die Reaktion \ce{2NH3 + 3I2 <-> 6HI + N2} bei $700K$. (Alle Reaktionspartner sind im gasf�rmigen Zustand).
		
		}
		
	\end{karte}
%%%%%%%%%%%%%%%%%%%%%%%%%%%%%%%%%%%%%

%%%%%%%%%%%%%%%%%%%%%%%%%%%%%%%%%%%%%
	\begin{karte}{
		
		Schwefeltrioxid zersetzt sich bei hoher Temperatur in einem geschlossenen Beh�lter gem��: \ce{2SO3 <-> 2SO2 + O2} (Alle Reaktionspartner sind im gasf�rmigen Zustand). Ein Gef�� wird bei $1000K$ mit \ce{SO_{3}} bei einem Partialdruck von $0,500atm$ gef�llt. Im Gleichgewicht ist der Partialdruck von \ce{SO3} $0,200atm$. Berechnen sie den Wert f�r $K_{p}$ bei $1000K$.
		
		}
		
	\end{karte}
%%%%%%%%%%%%%%%%%%%%%%%%%%%%%%%%%%%%%

%%%%%%%%%%%%%%%%%%%%%%%%%%%%%%%%%%%%%
	\begin{karte}{
		
		Bei 1000 K ist der Wert f�r $K_{p}$ der Reaktion \\
		\ce{2SO3 <->2SO2 + O2}\\
		gleich $0,338$. Sagen sie vorher welche Reaktion abl�uft, wenn ein Gemisch mit den Anfangspartialdr�cken von \\
		$p_{\ce{SO3}}=0,16atm$; \\
		$p_{\ce{SO2}}=0.41atm$; \\
		$p_{\ce{O2}}=2.5atm$ betrachtet wird.
		
		}
		
	\end{karte}
%%%%%%%%%%%%%%%%%%%%%%%%%%%%%%%%%%%%%

%%%%%%%%%%%%%%%%%%%%%%%%%%%%%%%%%%%%%
	\begin{karte}{
		
		Schreiben sie den Gleichgewichtsausdruck f�r das Gleichgewicht: \ce{C_{(s)} + CO2_{(g)} <-> 2CO_{(g)}}.
		Die unten angef�hrte Tabelle zeigt die Molprozente von \ce{CO2} und \ce{CO} bei einem Gesamtdruck von 1 atm f�r mehrere Temperaturen.
		Berechnen sie den Wert von Kp bei jeder Temperatur. 
		Ist die Reaktion exotherm oder endotherm? 
		Begr�nden sie Ihre Antwort. (R = 0.0821 L atm/Mol K).

		\begin{tabular}{ccc}
			\hline 
			$Temperatur$ & $CO_{2}$ & $CO$ \tabularnewline
			$\celsius $ & $Mol\%$ & $Mol\%$ \tabularnewline
			\hline 
			\hline 
			$850$ & $6,23$ & $93,77$ \tabularnewline
			$950$ & $1,32$ & $98,68$ \tabularnewline
			$1050$ & $0,37$ & $99,63$ \tabularnewline
			\hline 
		\end{tabular}
		
		}
		
	\end{karte}
%%%%%%%%%%%%%%%%%%%%%%%%%%%%%%%%%%%%%

%%%%%%%%%%%%%%%%%%%%%%%%%%%%%%%%%%%%%
	\begin{karte}{
		
		F�r das Gleichgewicht \ce{PCl5 <-> PCl3 + Cl2} (Alle Reaktionspartner sind im gasf�rmigen Zustand) betr�gt Kp @ 500K 0.497. \\
		Eine Gasflasche wird bei einem Anfangsdruck von 1.66 atm gef�llt.\\
		Was sind die Gleichgewichtsdr�cke f�r die drei Gase bei dieser Temperatur?
		
		}
		
	\end{karte}
%%%%%%%%%%%%%%%%%%%%%%%%%%%%%%%%%%%%%

