%%%%%%%%%%%%%%%%%%%%%%%%%%%%%%%%%%%%%%%%%%%%%%%%%%%%%%%%%%%%%%%%%%%%%%%%%%
%%%%%%%%%%%%%%%%%%%%%%%%%%%%%%%%%%%%%%%%%%%%%%%%%%%%%%%%%%%%%%%%%%%%%%%%%%
\section*{THERMOCHEMIE}
%%%%%%%%%%%%%%%%%%%%%%%%%%%%%%%%%%%%%%%%%%%%%%%%%%%%%%%%%%%%%%%%%%%%%%%%%%
%%%%%%%%%%%%%%%%%%%%%%%%%%%%%%%%%%%%%%%%%%%%%%%%%%%%%%%%%%%%%%%%%%%%%%%%%%

\newpage

%%%%%%%%%%%%%%%%%%%%%%%%%%%%%%%%%%%%%
	\begin{karte}{
		Methylhydrazin, ein Raketentreibstoff, verbrennt nach der Gleichung:\\
		\ce{2C_{6}N_{2(l)} +5O_{2(g)} -> 2N_{2(g)} +2CO_{2(g)} +6H_{2}O_{(l)}}\\
		Wenn 4 g Methylhydrazin verbrannt werden, steigt die Temperatur eines
		Bombenkalorimeters von 25.00�C auf 39.50�C
		an. F�r das Kalorimeter wurde eine W�rmekapazit�t von 7.794 kJ/�C
		bestimmt. \\
		Wie gro� ist die Reaktionsw�rme von einem Mol \ce{CH_{6}N_{2}}?
		}
		\begin{enumerate}
			\item 
				$E_{4g}		=7,794\frac{kJ}{�C} \cdot (39,50�C - 25,00�C)	=113,013kJ$ \\
				$M_{\ce{CH6N2}}	=12+(6\cdot1)+(2\cdot14)					=46\frac{g}{Mol}$\\
				$E_{Mol}		=\frac{E_{4g}}{4}\cdot M_{\ce{CH6N2}}	=\frac{113kJ}{4}\cdot46	=1300kJ$
		\end{enumerate}
	\end{karte}
%%%%%%%%%%%%%%%%%%%%%%%%%%%%%%%%%%%%%

%%%%%%%%%%%%%%%%%%%%%%%%%%%%%%%%%%%%%
	\begin{karte}{
		Berechnen sie die Standardenthalpie�nderung der Verbrennung von 1
		Mol Benzol zu \ce{CO_{2}} und \ce{H_{2}O} und formulieren Sie die Reaktionsgleichung.\\
		Wie viel Energie wird beim Verbrennen von 1.00 g Benzol frei? \\
		$\Delta H�_{f}(\ce{CO_{2}})=-393.5kJ$;\\
		$\Delta H�_{f}(\ce{H_{2}O})=-285.8kJ$;\\
		$\Delta H�_{f}($Benzol$)=49.0kJ$.
		}
		
		\begin{enumerate}
			\item 
				Benzol: \ce{C6H6} \\
				\ce{2C6H6 + 15O2 -> 12CO2 + 6H2O}
			
			\item
				$\Delta H	=\sum \Delta H�_{f} (Produkte) - \sum \Delta H�_{f} (Edukte)$ \\
				$\Delta H_{\ce{2C6H6}}	=12\cdot(-393,5) + 6\cdot(-285,8) - (2\cdot49,0)	=-6535kJ$ \\
			$\Delta H_{\ce{C6H6}}	=\frac{-6535}{2}							=-3268\frac{kJ}{Mol}$
			
			\item
			$M_{\ce{C6H6}}	=6\cdot12+6\cdot1	=78\frac{g}{Mol}$\\
			$E_{\ce{C6H6}}				=\frac{-3268}{78}	=-41,9\frac{kJ}{g}$
		\end{enumerate}
		
	\end{karte}
%%%%%%%%%%%%%%%%%%%%%%%%%%%%%%%%%%%%%




%%%%%%%%%%%%%%%%%%%%%%%%%%%%%%%%%%%%%
	\begin{karte}{
		Berechnen sie die Standardenthalpie�nderung der Verbrennung von 1
		Mol Methan zu \ce{CO2} und \ce{H2O} und formulieren Sie die Reaktionsgleichung.\\
		Wie viel Energie wird beim Verbrennen von 19.00 g Methan frei? \\
		$\Delta H�_{f}(\ce{CO_{2}})=-393.5kJ$;\\
		$\Delta H�_{f}(\ce{H_{2}O})=-285.8kJ$;\\
		$\Delta H�_{f}(Methan)=-74.80kJ$.
		}
	\begin{enumerate}
	
		\item 
		Methan: \ce{CH4} \\
		\ce{CH4 + 2O2 -> CO2 + 2H2O}
		
		\item
		$\Delta H	=\sum \Delta H�_{f} (Produkte) - \sum \Delta H�_{f} (Edukte)$ \\
		$\Delta H_{\ce{CH4}}	=-393,5 + 2\cdot(-285,8) - (-74,8)	=-890,3kJ$ 
		
		\item
		$M_{\ce{CH4}}	=12+4\cdot1	=16\frac{g}{Mol}$\\
		$E_{\ce{CH4}}				=\frac{-890,3}{16}		=-55,6\frac{kJ}{g}$ \\
		$E						=-55,6\frac{kJ}{g}\cdot19g	=1057kJ$
		
	\end{enumerate}
	\end{karte}
%%%%%%%%%%%%%%%%%%%%%%%%%%%%%%%%%%%%%





%%%%%%%%%%%%%%%%%%%%%%%%%%%%%%%%%%%%%
	\begin{karte}{
		Berechnen sie die Standardenthalpie�nderung der Verbrennung von 1
			Mol Graphit zu \ce{CO_{2}} und formulieren Sie die Reaktionsgleichung.\\
			Wie viel Energie wird beim Verbrennen von 13 kg Graphit frei? \\
			$\Delta H�_{f}(\ce{CO_{2}})=-393.5kJ$;
		}
		
			\begin{enumerate}
			
				\item 
				Graphit: \ce{C} ;-) \\
				\ce{C + O2 -> CO2} 
				
				\item
				$\Delta H	=\sum \Delta H�_{f} (Produkte) - \sum \Delta H�_{f} (Edukte)$ \\
				$\Delta H_{\ce{C}}	=-393,5kJ$
				
				\item
				$M_{\ce{C}}	=12\frac{g}{Mol}$ \\
				$E_{\ce{C}}	=\frac{-393,5}{12}	=-32,8\frac{kJ}{g}$\\
				$E			=-32,8\frac{kJ}{g} \cdot 13000g	=-426291, 7kJ$
		
			\end{enumerate}
		
	\end{karte}
%%%%%%%%%%%%%%%%%%%%%%%%%%%%%%%%%%%%%




%%%%%%%%%%%%%%%%%%%%%%%%%%%%%%%%%%%%%
	\begin{karte}{
		Die Standardenthalpie�nderung der Ethanol-Verbrennung betr�gt -1367 kJ.
			Formulieren Sie die Reaktionsgleichung und berechnen Sie die Standardbildungsenthalpie
			von Ethanol.\\
			$\Delta H�_{f}(\ce{CO_{2}})=-393.5kJ$;\\
			$\Delta H�_{f}(\ce{H_{2}O})=-285.8kJ$
		}
		
		
			\begin{enumerate}
		
				\item 
				Ethanol: \ce{C2H6O} \\
				\ce{C2H6O + 3O2 -> 2CO2 + 3H2O}
				
				\item
				$\Delta H	=\sum \Delta H�_{f} (Produkte) - \sum \Delta H�_{f} (Edukte)$ \\
				$\Rightarrow \sum \Delta H�_{f} (Edukte) = \sum \Delta H�_{f} (Produkte) - \Delta H$ \\
				$\Delta H�_{f} (\ce{C2H6O})	=2\cdot(-393,5kJ) + 3\cdot(-285,8kJ) - (-1367kJ)	=-277,4kJ$
			
			\end{enumerate}
		
	\end{karte}
%%%%%%%%%%%%%%%%%%%%%%%%%%%%%%%%%%%%%






