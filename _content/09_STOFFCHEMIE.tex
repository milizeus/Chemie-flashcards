%!TEX root = ../chemie.tex




\newpage

\chapter{STOFFCHEMIE}

\begin{enumerate}

	%%%%%%%%%%%%%%%%%%%%%%%%%%%%%%%%%%%%%%%%%%%%%%%%%%%%%%%%
	
	\item Beschreiben sie Eigenschaften und Verwendung von Schwefel und Selen.
	
	\begin{enumerate}
		\item a
	\end{enumerate}

	%%%%%%%%%%%%%%%%%%%%%%%%%%%%%%%%%%%%%%%%%%%%%%%%%%%%%%%%
	\item Beschreiben sie die Herstellung und Verwendung von Stickstoff.
	
	\begin{enumerate}
		\item a
	\end{enumerate}

	%%%%%%%%%%%%%%%%%%%%%%%%%%%%%%%%%%%%%%%%%%%%%%%%%%%%%%%%

	\item Beschreiben sie Vorkommen und Herstellung von Silizium.
	
	\begin{enumerate}
		\item a
	\end{enumerate}

	%%%%%%%%%%%%%%%%%%%%%%%%%%%%%%%%%%%%%%%%%%%%%%%%%%%%%%%%

	\item Beschreiben sie die Herstellung von Stahl. Zeichnen sie drei Isomere
	der Summenformel \ce{C5H12} und geben sie einen chemischen Namen f�r jede
	
	\begin{enumerate}
		\item a
	\end{enumerate}

	%%%%%%%%%%%%%%%%%%%%%%%%%%%%%%%%%%%%%%%%%%%%%%%%%%%%%%%%

	\item Zeichnen sie die Strukturformeln des cis- und des trans-Isomers von
	3- Penten-1-ol. Kann bei Cyclopenten eine cis-trans Isomerie vorliegen?
	Erkl�ren sie ihre Antwort.
	
	\begin{enumerate}
		\item a
	\end{enumerate}

	%%%%%%%%%%%%%%%%%%%%%%%%%%%%%%%%%%%%%%%%%%%%%%%%%%%%%%%%

	\item Beschreiben Sie die sechs Kohlenwasserstofffraktionen der Erd�ldestillation,
	deren Siedepunktsbereiche und deren Verwendung.
	
	\begin{enumerate}
		\item a
	\end{enumerate}

	%%%%%%%%%%%%%%%%%%%%%%%%%%%%%%%%%%%%%%%%%%%%%%%%%%%%%%%%

	\item Beschreiben Sie die molekulare Grundlage unserer Sehf�higkeit.
	
	\begin{enumerate}
		\item a
	\end{enumerate}

	%%%%%%%%%%%%%%%%%%%%%%%%%%%%%%%%%%%%%%%%%%%%%%%%%%%%%%%%

	\item Definieren Sie Chiralit�t und zeichnen Sie ein beliebiges chirales
	Molek�l.
	
	\begin{enumerate}
		\item a
	\end{enumerate}

	%%%%%%%%%%%%%%%%%%%%%%%%%%%%%%%%%%%%%%%%%%%%%%%%%%%%%%%%

	\item Zeichnen Sie drei beliebige nat�rliche Aminos�uren und beschreiben
	Sie die Natur der Peptidbindung.
	
	\begin{enumerate}
		\item a
	\end{enumerate}

	%%%%%%%%%%%%%%%%%%%%%%%%%%%%%%%%%%%%%%%%%%%%%%%%%%%%%%%%

	\item Zeichnen Sie die Wiederholeinheit von Polyethylen, Polystyrol und
	Nylon 6,6. Geben sie je zwei Anwendungsgebiete dieser Kunststoffe
	an.

	\begin{enumerate}
		\item a
	\end{enumerate}
		
	%%%%%%%%%%%%%%%%%%%%%%%%%%%%%%%%%%%%%%%%%%%%%%%%%%%%%%%%

\end{enumerate}



%%%%%%%%%%%%%%%%%%%%%%%%%%%%%%%%%%%%%%%%%%%%%%%%%%%%%%%%


