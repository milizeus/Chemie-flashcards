%%%%%%%%%%%%%%%%%%%%%%%%%%%%%%%%%%%%%%%%%%%%%%%%%%%%%%%%%%%%%%%%%%%%%%%%%%
%%%%%%%%%%%%%%%%%%%%%%%%%%%%%%%%%%%%%%%%%%%%%%%%%%%%%%%%%%%%%%%%%%%%%%%%%%
\section*{STOFFCHEMIE}
%%%%%%%%%%%%%%%%%%%%%%%%%%%%%%%%%%%%%%%%%%%%%%%%%%%%%%%%%%%%%%%%%%%%%%%%%%
%%%%%%%%%%%%%%%%%%%%%%%%%%%%%%%%%%%%%%%%%%%%%%%%%%%%%%%%%%%%%%%%%%%%%%%%%%

%%%%%%%%%%%%%%%%%%%%%%%%%%%%%%%%%%%%%
	\begin{karte}{
		
		Beschreiben sie Eigenschaften und Verwendung von Schwefel und Selen.
		
		}
		
		Antwort
		
	\end{karte}
%%%%%%%%%%%%%%%%%%%%%%%%%%%%%%%%%%%%%

%%%%%%%%%%%%%%%%%%%%%%%%%%%%%%%%%%%%%
	\begin{karte}{
		
		Beschreiben sie die Herstellung und Verwendung von Stickstoff.
		
		}
		
		Antwort
		
	\end{karte}
%%%%%%%%%%%%%%%%%%%%%%%%%%%%%%%%%%%%%

%%%%%%%%%%%%%%%%%%%%%%%%%%%%%%%%%%%%%
	\begin{karte}{
		
		Beschreiben sie Vorkommen und Herstellung von Silizium.
		
		}
		
		Antwort
		
	\end{karte}
%%%%%%%%%%%%%%%%%%%%%%%%%%%%%%%%%%%%%

%%%%%%%%%%%%%%%%%%%%%%%%%%%%%%%%%%%%%
	\begin{karte}{
		
		Beschreiben sie die Herstellung von Stahl. Zeichnen sie drei Isomere der Summenformel \ce{C5H12} und geben sie einen chemischen Namen f�r jede
		
		}
		
		Antwort
		
	\end{karte}
%%%%%%%%%%%%%%%%%%%%%%%%%%%%%%%%%%%%%

%%%%%%%%%%%%%%%%%%%%%%%%%%%%%%%%%%%%%
	\begin{karte}{
		
		Zeichnen sie die Strukturformeln des cis- und des trans-Isomers von 3- Penten-1-ol. Kann bei Cyclopenten eine cis-trans Isomerie vorliegen?
		Erkl�ren sie ihre Antwort.
		
		}
		
		Antwort
		
	\end{karte}
%%%%%%%%%%%%%%%%%%%%%%%%%%%%%%%%%%%%%

%%%%%%%%%%%%%%%%%%%%%%%%%%%%%%%%%%%%%
	\begin{karte}{
		
		Beschreiben Sie die sechs Kohlenwasserstofffraktionen der Erd�ldestillation, deren Siedepunktsbereiche und deren Verwendung.
		
		}
		
		Antwort
		
	\end{karte}
%%%%%%%%%%%%%%%%%%%%%%%%%%%%%%%%%%%%%

%%%%%%%%%%%%%%%%%%%%%%%%%%%%%%%%%%%%%
	\begin{karte}{
		
		Beschreiben Sie die molekulare Grundlage unserer Sehf�higkeit.
		
		}
		
		Antwort
		
	\end{karte}
%%%%%%%%%%%%%%%%%%%%%%%%%%%%%%%%%%%%%

%%%%%%%%%%%%%%%%%%%%%%%%%%%%%%%%%%%%%
	\begin{karte}{
		
		Definieren Sie Chiralit�t und zeichnen Sie ein beliebiges chirales Molek�l.
		
		}
		
		Antwort
		
	\end{karte}
%%%%%%%%%%%%%%%%%%%%%%%%%%%%%%%%%%%%%

%%%%%%%%%%%%%%%%%%%%%%%%%%%%%%%%%%%%%
	\begin{karte}{
		
		Zeichnen Sie drei beliebige nat�rliche Aminos�uren und beschreiben Sie die Natur der Peptidbindung.
		
		}
		
		Antwort
		
	\end{karte}
%%%%%%%%%%%%%%%%%%%%%%%%%%%%%%%%%%%%%

%%%%%%%%%%%%%%%%%%%%%%%%%%%%%%%%%%%%%
	\begin{karte}{
		
		Zeichnen Sie die Wiederholeinheit von Polyethylen, Polystyrol und Nylon 6,6. Geben sie je zwei Anwendungsgebiete dieser Kunststoffe an.
		
		}
		
		Antwort
		
	\end{karte}
%%%%%%%%%%%%%%%%%%%%%%%%%%%%%%%%%%%%%

